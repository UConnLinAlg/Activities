
  \documentclass{ximera}
% augmented matrix
\newenvironment{amatrix}[1]{%
	\left[\begin{array}{@{}*{#1}{c}|c@{}}
	}{%
\end{array}\right]
}


\newcommand{\RR}{\bf R}
\newcommand{\adj}{\mathop{\rm adj}}
\newcommand{\degrees}{$^{\circ}$}
\newcommand{\Col}{\mathop{\rm Col}}
\newcommand{\diag}{\mathop{\rm diag}}
\newcommand{\Inn}{\mathop{\rm Inn}}
\newcommand{\Nul}{\mathop{\rm Nul}}
\newcommand{\rank}{\mathop{\rm rank}}
\newcommand{\rk}{\mathop{\rm rk}}
\newcommand{\Row}{\mathop{\rm Row}}
%% \def Doesn't work with Ximera, MathJax
%\def\Span{\mathop{\rm Span}}
\newcommand{\Span}{\mathop{\rm Span}}
\newcommand{\spec}{\mathop{\rm spec}}
\newcommand{\tr}{\mathop{\rm tr}}
\newcommand{\vol}{\mathop{\rm vol}}

%  \input{preamble.tex}
  \begin{document}
  	\title{Lay 5.4  \hfill Math 2210Q} 
  	  		    %----------------------------------------------------------------
  	  		    \begin{question} True/False: Let $T:V\to W$ be a linear transformation. The matrix for $T$ relative to the bases $\mathcal{B}$ and $\mathcal{C}$ for $V$ and $W$ respectively is given by:
  	  		    	$$M =\begin{bmatrix} [T(\vec{b}_1)]_{\mathcal{C}}& [T(\vec{b}_2)]_{\mathcal{C}} &\cdots& [T(\vec{b}_n)]_{\mathcal{C}}\end{bmatrix}$$
  	  		    	where $\mathcal{B} = \{\vec{b}_1,\vec{b}_2, \dots, \vec{b}_n  \}$.
  	  		    	
  	  		    	\begin{multipleChoice}
  	  		    		\choice[correct]{True}
  	  		    		\choice{False}
  	  		    	\end{multipleChoice}
  	  		    	
  	  		    \end{question}	
  	  		    %----------------------------------------------------------------
  	  		     
  	  		      \begin{question} True/False: Let $T:V\to W$ be a linear transformation. Let $\mathcal{B}$ and $\mathcal{C}$ be bases for $V$ and $W$ respectively. Let $M$ be the matrix for $T$ relative to $\mathcal{B}$ and $\mathcal{C}$. Then which of the following equations is true? 
  	  		      	
  	  		      	\begin{multipleChoice}
  	  		      		\choice[correct]{$ [T(\vec{x})]_\mathcal{C} = M[\vec{x}]_\mathcal{B}$}
  	  		      		\choice{$ [T(\vec{x})]_\mathcal{B} = M[\vec{x}]_\mathcal{C}$}
  	  		      	\end{multipleChoice}
  	  		      	
  	  		      \end{question}	
  	  		      %----------------------------------------------------------------
  	  		 \begin{question} Let$\mathcal{B} = \{\vec{b}_1, \vec{b}_2\}$ and $\mathcal{C} = \{\vec{c}_1, \vec{c}_2\}$ be bases for the vector spaces $V$ and $W$ respectively. Let  $T:V\to W$ be a linear transformation. Given the equations below, find the matrix for $T$ relative to $\mathcal{B}$ and $\mathcal{C}$.
  	  		 	$$ T(\vec{b}_1) = 4\vec{c}_1 +2\vec{c}_2 \hspace{10pt} T(\vec{b}_2) = -3\vec{c}_2$$
  	  		 	
  	  		 	$M = \begin{bmatrix} \answer{4} & \answer{0}\\ \answer{2} &\answer{-3}\end{bmatrix}$
  	  		 	
  	  		 	\end{question}
  	  		      %----------------------------------------------------------------
  	  		      	 \begin{question} Let $\mathcal{B} = \{\vec{b}_1, \vec{b}_2,\vec{b}_3\}$ and $\mathcal{C} = \{\vec{c}_1, \vec{c}_2\}$ be bases for the vector spaces $V$ and $W$ respectively. Let  $T:V\to W$ be a linear transformation. Given the equations below, find the matrix for $T$ relative to $\mathcal{B}$ and $\mathcal{C}$.
  	  		      	 	$$ T(\vec{b}_1) = -\vec{c}_1 +2\vec{c}_2 \hspace{10pt} T(\vec{b}_2) = \vec{c}_1+\vec{c}_2 \hspace{10pt} T(\vec{b}_3) = 5\vec{c}_1$$
  	  		      	 	
  	  		      	 	$M = \begin{bmatrix} \answer{-1} & \answer{1}& \answer{5}\\ \answer{2} &\answer{1}&\answer{0} \end{bmatrix}$
  	  		      	 	
  	  		      	 \end{question}
  	  		      	 %----------------------------------------------------------------
  	  		      	  \begin{question} Let $\mathcal{B}$ be a basis for some vector space $V$. If the linear transformation $T: V\to V$ sends vectors written with respect to the basis $\mathcal{B}$ to vectors written with respect to the basis $\mathcal{B}$, then the matrix for $T$ relative to $\mathcal{B}$  (or the $\mathcal{B}$-matrix for $T$) satisfies: 
  	  		      	  	$$[T(\vec{x})]_\mathcal{B} =[T]_\mathcal{B}[\vec{x}]_\mathcal{B}$$
  	  		      	  	\begin{multipleChoice}
  	  		      	  		\choice[correct]{True}
  	  		      	  		\choice{False}
  	  		      	  	\end{multipleChoice}
  	  		      	  	
  	  		      	  \end{question}
  	  		      	  %----------------------------------------------------------------
  	  		      	   	  \begin{question}True/False. Suppose $A=PDP^{-1}$ where $D$ is a diagonal $n\times n$ matrix. If $\mathcal{B}$ is the basis for $\RR^n$ formed fromt he columns of $P$, then $D$ is the $\mathcal{B}$-matrix for the transformation $\vec{x} \mapsto A\vec{x}$.\\
  	  		      	   	  	
  	  		      	   	  	\begin{multipleChoice}
  	  		      	   	  		\choice[correct]{True}
  	  		      	   	  		\choice{False}
  	  		      	   	  		\end{multipleChoice}
  	  		      	   	  		
  	  		      	   	  		\begin{hint}
  	  		      	   	  			See the diagonal matrix representation theorem on page 291 of Lay.
  	  		      	   	  			\end{hint}
  	  		      	   	  	
  	  		      	   	  \end{question}
  	  		      	   	  %----------------------------------------------------------------
  	  		      	   	   \begin{question} Suppose $A = PDP^{-1}$ where $P,D, P^{-1}$ are given below. Let the linear transformation $T:\RR^3 \to \RR^3$ be defined by $T(\vec{x}) = A\vec{x}$. Which of the following gives a basis $\mathcal{B}$ for $\RR^3$ with the property that $[T]_\mathcal{B}$ is diagonal.
  	  		      	   	   	
  	  		      	   	   	$$P = \begin{bmatrix} 3 & 0 &-1\\ 0& 1&-3 \\1&0&0\end{bmatrix} \hspace{10pt} 
  	  		      	   	   	D = \begin{bmatrix} 3 & 0 &0\\ 0& 4&0 \\0&0&3\end{bmatrix} \hspace{10pt} 
  	  		      	   	   P^{-1} =	\begin{bmatrix} 0 & 0 &1\\ -3& 1&9 \\-1&0&3\end{bmatrix} $$
  	  		      	   	   	
  	  		      	   	   	\begin{multipleChoice}
  	  		      	   	   		\choice{$\begin{bmatrix} 3\\0\\0\end{bmatrix},\begin{bmatrix} 0\\4\\0\end{bmatrix}, \begin{bmatrix} 0\\0\\3\end{bmatrix} $}
  	  		      	   	   			\choice{$\begin{bmatrix} 3\\-3\\0\end{bmatrix},\begin{bmatrix} 0\\4\\0\end{bmatrix}, \begin{bmatrix} 0\\9\\3\end{bmatrix} $}
  	  		      	   	 
  	  		      	   	   					\choice[correct]{$\begin{bmatrix} 3\\0\\1\end{bmatrix},\begin{bmatrix} 0\\1\\0\end{bmatrix}, \begin{bmatrix} -1\\-3\\0\end{bmatrix} $}
  	  		      	   	   	\end{multipleChoice}
  	  		      	   	   	
  	  		      	   	   	\begin{hint}
  	  		      	   	   		See the diagonal matrix representation theorem on page 291 of Lay.
  	  		      	   	   	\end{hint}
  	  		      	   	   	
  	  		      	   	   \end{question}
  	  		      	   	   %----------------------------------------------------------------
  \end{document}
