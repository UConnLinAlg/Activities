
  \documentclass{ximera}
% augmented matrix
\newenvironment{amatrix}[1]{%
	\left[\begin{array}{@{}*{#1}{c}|c@{}}
	}{%
\end{array}\right]
}


\newcommand{\RR}{\bf R}
\newcommand{\adj}{\mathop{\rm adj}}
\newcommand{\degrees}{$^{\circ}$}
\newcommand{\Col}{\mathop{\rm Col}}
\newcommand{\diag}{\mathop{\rm diag}}
\newcommand{\Inn}{\mathop{\rm Inn}}
\newcommand{\Nul}{\mathop{\rm Nul}}
\newcommand{\rank}{\mathop{\rm rank}}
\newcommand{\rk}{\mathop{\rm rk}}
\newcommand{\Row}{\mathop{\rm Row}}
\newcommand{\proj}{\mathop{\rm proj}}
%% \def Doesn't work with Ximera, MathJax
%\def\Span{\mathop{\rm Span}}
\newcommand{\Span}{\mathop{\rm Span}}
\newcommand{\spec}{\mathop{\rm spec}}
\newcommand{\tr}{\mathop{\rm tr}}
\newcommand{\vol}{\mathop{\rm vol}}

%  \input{preamble.tex}
  \begin{document}
  	\title{Lay 6.4  \hfill Math 2210Q} 
  	  		    %----------------------------------------------------------------
  	  		    \begin{question} 
  	  		    	Let $\{\vec{x}_1, \vec{x}_2\}$ be a basis for a subspace $W$. Use the Gram-Schmidt process to find and orthogonal basis $\{\vec{v}_1,\vec{v}_2\}$ for $W$.
  	  		    $$\vec{x}_1 = \begin{bmatrix} 2\\1\\4\end{bmatrix}, \hspace{10pt} \vec{x}_2 = \begin{bmatrix} -1\\5\\0\end{bmatrix}$$
  	  		    
  	  		    $\vec{v}_1 = \begin{bmatrix} \answer{2}\\\answer{1}\\\answer{4}\end{bmatrix}$\vspace{10pt}\\
  	  		    $\vec{v}_2 = \frac{1}{7}\begin{bmatrix} \answer{-9}\\\answer{34}\\\answer{-4}\end{bmatrix}$\vspace{10pt}\\
  	  		    
  	  		    	\end{question}
  	  		    	%----------------------------------------------------------------
%%% TRKEE 24 JUNE 2017: Wrong answer in 2nd vector, when corrected makes 3rd vector quite painful
  	  		    	 % \begin{question} 
  	  		    	 % 	Let $\{\vec{x}_1, \vec{x}_2,\vec{x}_3\}$ be a basis for a subspace $W$. Use the Gram-Schmidt process to find and orthogonal basis $\{\vec{v}_1,\vec{v}_2,\vec{v}_3\}$ for $W$.
  	  		    	 % 	$$\vec{x}_1 = \begin{bmatrix} 3\\1\\2\\1\end{bmatrix}, \hspace{10pt} \vec{x}_2 = \begin{bmatrix} 1\\4\\0\\1\end{bmatrix},\hspace{10pt} \vec{x}_3 = \begin{bmatrix} 1\\0\\0\\1\end{bmatrix}$$
  	  		    	 	
  	  		    	 % 	$\vec{v}_1 = \begin{bmatrix} \answer{3}\\\answer{1}\\\answer{2}\\\answer{1}\end{bmatrix}$\vspace{10pt}\\
  	  		    	 % 	$\vec{v}_2 = \frac{1}{15}\begin{bmatrix} \answer{-6}\\\answer{52}\\\answer{-16}\\\answer{7}\end{bmatrix}$\vspace{10pt}\\
  	  		    	 % 		$\vec{v}_3 = \frac{1}{45}\begin{bmatrix} \answer{4}\\\answer{-32}\\\answer{-24}\\\answer{28}\end{bmatrix}$\vspace{10pt}\\
  	  		    	 % 	Is $\{\vec{v}_1,\vec{v}_2,\vec{v}_3\}$ an orthonormal basis for $W$?
  	  		    	 % 	\begin{multipleChoice}
  	  		    	 % 		\choice{Yes}
  	  		    	 % 		\choice[correct]{No}
  	  		    	 % 		\end{multipleChoice}
  	  		    	 % \end{question}
  	  		             %----------------------------------------------------------------
  	  		              \begin{question} 
  	  		              Suppose using the Gram-Schmidt process, you found that the set $\{\vec{v}_1,\vec{v}_2\}$ is an orthogonal basis for some subspace $W$. Find an orthonormal basis $\{\vec{w}_1,\vec{w}_2\}$ i for $W$,
  	  		              	$$\vec{v}_1 = \begin{bmatrix} 1\\1\\2\end{bmatrix}, \hspace{10pt} \vec{v}_2 = \begin{bmatrix} 10\\0\\-5\end{bmatrix}$$
  	  		              	
  	  		              	$\vec{w}_1 = \frac{1}{\sqrt{\answer{6}}}\begin{bmatrix} \answer{1}\\\answer{1}\\\answer{2}\end{bmatrix}$\vspace{10pt}\\
  	  		              	$\vec{w}_2 = \frac{1}{\sqrt{\answer{5}}}\begin{bmatrix} \answer{2}\\\answer{0}\\\answer{-1}\end{bmatrix}$\vspace{10pt}\\
  	  		              
  	  		              \end{question}
  	  		              %----------------------------------------------------------------
  	  		                \begin{question} 
  	  		                	Suppose the matrix $A$ below has three pivot columns. 
  	  		                	
  	  		                	$$A = \begin{bmatrix}0&-2&3\\1&2&1\\1&2&4\\0&1&-1    \end{bmatrix}$$
  	  		                	
  	  		                	First, find a basis for $\Col A$.\vspace{10pt}\\
  	  		                	
  	  		                	$\left\{  	\begin{bmatrix} \answer{0}\\\answer{1}\\\answer{1}\\\answer{0}\end{bmatrix},  \begin{bmatrix} \answer{-2}\\\answer{2}\\\answer{2}\\\answer{1}\end{bmatrix} , \begin{bmatrix}\answer{3}\\\answer{1}\\\answer{4}\\\answer{-1}\end{bmatrix}    \right\}$\vspace{10pt}\\
  	  		                	
  	  		                	
  	  		               Now, use the Gram-Schmidt process to find an orthogonal basis for $\Col A$.\\
  	  		                	
  	  		                		$\left\{   \begin{bmatrix} \answer{0}\\\answer{1}\\\answer{1}\\\answer{0}\end{bmatrix}, 
  	  		                		\begin{bmatrix} \answer{-2}\\\answer{0}\\\answer{0}\\\answer{1}\end{bmatrix},  
  	  		                		\begin{bmatrix} \frac{\answer{45}}{13}\\\frac{\answer{-51}}{26}\\\frac{\answer{27}}{26}\\\frac{\answer{-16}}{13}\end{bmatrix},                        \right\}$
  	  		                	
  	  		                \end{question}
  	  		                      %----------------------------------------------------------------
  	  		                      \begin{question} 
  	  		                       Let $A = \begin{bmatrix} 1&2&4\\ 3&4&10\\ 0&1&1\end{bmatrix}$. First, find a basis for $\Col A$.\vspace{10pt}\\
  	  		                      	
  	  		                      	$\left\{   \begin{bmatrix} \answer{1}\\\answer{3}\\\answer{0}\end{bmatrix},  
  	  		                      	\begin{bmatrix} \answer{2}\\\answer{4}\\\answer{1}\end{bmatrix}
  	  		                                       \right\}$\vspace{10pt}\\
  	  		                      	
  	  		                      	
  	  		                      	Now, use the Gram-Schmidt process to find an orthogonal basis for $\Col A$.\\
  	  		                      	
  	  		                      	$\left\{   \begin{bmatrix} \answer{1}\\\answer{3}\\\answer{0}\end{bmatrix},  
  	  		                      	\begin{bmatrix} \answer{1}/2\\\answer{-1}/2\\\answer{1}\end{bmatrix}\right\}$
  	  		                      	
  	  		                      	\begin{hint}
  	  		                      		
  	  		                      		The column space of $A$ is only two dimensional in this problem. How do you find a basis for $\Col A$?
  	  		                      		\end{hint}
  	  		                      \end{question}
  	  		                      %----------------------------------------------------------------
  \end{document}
