
  \documentclass{ximera}
% augmented matrix
\newenvironment{amatrix}[1]{%
	\left[\begin{array}{@{}*{#1}{c}|c@{}}
	}{%
\end{array}\right]
}


%\renewcommand{\vec}[1]{\mathbf{#1}}
\newcommand{\RR}{\bf R}
\newcommand{\adj}{\text{ adj }}

%  \input{preamble.tex}
  \begin{document}
  	\title{Lay 3.3  \hfill Math 2210Q} 

  %----------------------------------------------------------------
  \begin{question} For any $n\times n$ matrix $A$ and any vector $\vec{b} \in \RR^n$, $A_i(\vec{b})$ is the matrix obtained from $A$ be replacing column $i$ by $\vec{b}$. \\
  	\vspace{10pt}\\
  	Given that $A = \begin{bmatrix} 1& 7\\ -2 &-3\end{bmatrix}$ and $\vec{b} = \begin{bmatrix} 2\\ 9\end{bmatrix}$ compute the following.\\ \\
  	
  	$A_1(\vec{b}) =\begin{bmatrix}
  	\answer{2} &\answer{7}\\
  	\answer{9} & \answer{-3}
  	\end{bmatrix}$\vspace{10pt}\\
  	
  		$A_2(\vec{b}) =\begin{bmatrix}
  		\answer{1} &\answer{2}\\
  		\answer{-2} & \answer{9}
  		\end{bmatrix}$\\
  	
  
  	
  	\end{question}	
  %----------------------------------------------------------------
     \begin{question} Use Cramer's rule to determine the unique solution $\vec{x}$ to the equation: $$\begin{bmatrix} -4 & 1 \\ 2& 12 \end{bmatrix}\vec{x} = \begin{bmatrix} 3\\-1\end{bmatrix}.$$
     	
     	$x_1=-\dfrac{\answer{37}}{\answer{50}}$\vspace{10pt}\\
     	$x_2=\dfrac{\answer{1}}{\answer{25}}$ (reduce the fraction)\vspace{10pt}\\
   
     	
     	
     	
     \end{question}	
     %----------------------------------------------------------------
      \begin{question} Use Cramer's rule to determine the unique solution $\vec{x}$ to the equation: $$\begin{bmatrix} 1 & -5 \\ 7& -8 \end{bmatrix}\vec{x} = \begin{bmatrix} 1\\1\end{bmatrix}.$$
      	
      	$x_1=-\dfrac{\answer{1}}{\answer{9}}$ (reduce the fraction)\vspace{10pt}\\
      	$x_2=-\dfrac{\answer{2}}{\answer{9}}$ (reduce the fraction)\vspace{10pt}\\
      	
      	
      	
      	
      \end{question}	
      %----------------------------------------------------------------
       \begin{question} For which of the following matrices can you use cramer's rule to find the solution to $A\vec{x} = \vec{b}$ for some vector $\vec{b}$ of the appropriate dimension? 
       	
       	\begin{multipleChoice}
       		\choice {$A = \begin{bmatrix} 1 &7\\ 3& 4\\ 0&7\end{bmatrix}$ }\vspace{5pt}\\
       			\choice[correct]{ $A = \begin{bmatrix} 1 &-9\\ 0& 1\end{bmatrix}$} \vspace{5pt}\\
       			\choice {$A = \begin{bmatrix} 1 &7&2&5\\ 0&1&1& 4\\ 0&0&0&-7\\0&0&0&5\end{bmatrix}$} \}vspace{5pt}\\
       		\end{multipleChoice}
       	
       	
     
       \end{question}	
      %----------------------------------------------------------------  
        \begin{question} For which of the following matrices can you use cramer's rule to find the solution to $A\vec{x} = \vec{b}$ for some vector $\vec{b}$ of the appropriate dimension? 
        	
        	\begin{multipleChoice}
       	\choice[correct]{ $A = \begin{bmatrix} 1 &0 &2\\ 0&7& 1 \\0&3&1\end{bmatrix}$} \vspace{5pt}\\
       	\choice{$A = \begin{bmatrix} 2 &-6\\ -3& 9\end{bmatrix}$}\vspace{5pt}\\	
       	  		\end{multipleChoice}
       	  		
       	  		
       	  		
       	  	\end{question}	
       %----------------------------------------------------------------
          \begin{question} Theorem: Let $A$ be an invertible $n\times n$ matrix. Then $$A^{-1} = \dfrac{1}{\det A}  \adj  A.$$
          	
          Let $A = \begin{bmatrix} 2& 4& 7\\ 1& 3&1\\ -1&-1&2\end{bmatrix}$. Compute the following.\vspace{10pt}\\
          
          $\det A = \answer{16}$\vspace{10pt}\\
          
          $\adj A = \begin{bmatrix} 
          \answer{7} & \answer{15} &\answer{-17}\\
          \answer{3} & \answer{11} &\answer{-5}\\
          \answer{2} & \answer{2} &\answer{2} \end{bmatrix}$\vspace{10pt}\\
          
      
          
          $A^{-1} =   \begin{bmatrix} 
          	\frac{\answer{7}}{\answer{16}} & -\frac{\answer{15}}{\answer{16}} &-\frac{\answer{17}}{\answer{16}}\\
          	\frac{\answer{3}}{\answer{16}}& \frac{\answer{11}}{\answer{16}} &-\frac{\answer{5}}{\answer{16}}\\
          	\frac{\answer{1}}{\answer{8} }& \frac{\answer{1}}{\answer{8}} &\frac{\answer{1}}{\answer{8}} \end{bmatrix}$
          	
          	
          \end{question}	
          %----------------------------------------------------------------
  \end{document}
